% !TEX TS-program = xelatex
% !TEX encoding = UTF-8 Unicode
% !Mode:: "TeX:UTF-8"

\documentclass{resume}
\usepackage{zh_CN-Adobefonts_external} % Simplified Chinese Support using external fonts (./fonts/zh_CN-Adobe/)
%\usepackage{zh_CN-Adobefonts_internal} % Simplified Chinese Support using system fonts
\usepackage{linespacing_fix} % disable extra space before next section
\usepackage{cite}

\begin{document}
\pagenumbering{gobble} % suppress displaying page number

\name{张恩宝}

\basicInfo{
  \email{zhangenbao@stumail.ysu.edu.cn} \textperiodcentered\ 
  \phone{(+86) 175-2685-6526} \textperiodcentered\ 
  \linkedin[北京]{https://www.linkedin.com/in/billryan8}}
 
\section{\faGraduationCap\  教育背景}
\datedsubsection{\textbf{燕山大学}}{2023.09 -- 2026.06}
\textit{研究生 }\ 专业:人工智能
\datedsubsection{\textbf{天津工业大学}}{2019.09 -- 2023.06}
\textit{本科  }\  专业:电子信息工程

\section{\faUsers\ 实习/项目经历}
\datedsubsection{\textbf{联想诺谛(北京)智能科技有限公司}}{2025年1月 -- 2025年5月}
\role{AI部:NLP算法实习生}
\datedsubsection{\textbf{工作内容}}
\begin{itemize}[itemsep=1ex, topsep=1ex, parsep=0.5ex, leftmargin=1em]
  \item 专注于大模型 CoT (Chain of Thought)推理技术,探索大模型\textbf{推理}及\textbf{涌现}能力。研究 SFT(监督微调)和 强化学习GRPO(基于规则的策略优化)等方法,改进优化大模型的推理生成效果。
    \item \textbf{合成 CoT 推理数据对大模型推理生成的影响:}
  {\linespread{1.2}\selectfont
  由模型 Qwen-72b、DeepSeekV3 生成 CoT 推理合成数据,合成流程包括 4 种搜索策略(探索、回溯、验证、校正)以确保数据具备推理特性和正确的结果。SFT 方法基于 LoRA 对 Qwen2.5-7b 进行优化;测试基于 SGLang 部署微调模型后进行评估。模型生成效果具备推理模式,但答案准确性相较基准模型略微降低。
  }
  \item \textbf{GRPO算法对大模型推理效果的影响:}基于 Unsloth 框架使用 GRPO 算法对 Qwen2.5 系列(7B、14B、32B)进行RL-GRPO 微调。奖励模版为<格式奖励>+<确定性奖励>,格式奖励依据数据集样式设置了具体奖励。经测试,模型生成效果具备分析推理模式(推理 Token 较为简洁)。Qwen2.5-32B 经 GRPO 优化后,答案准确率接近 QwQ-32B 模型,高于基准模型 3.51%
  \item \textbf{R1-Distill 推理数据 SFT + GRPO 对大模型推理效果的影响:}首先使用 R1 蒸馏数据 150k (中英混合) 对 Qwen2.5-7B 模型进行 LoRA 微调;基于 SFT 后的模型,引入 GRPO 进行强化学习优化。经测试,7B 模型输出具备推理模式(输出模板为:<think> + <answer>),答案准确性高于基准模型 1.57%

\end{itemize}

\datedsubsection{\textbf{Lenovo 数据综合分析(智能汇总)}}{2025年1月 -- 2025年5月}
\role{项目演示:\url{https://bcnfeafpcwdr.feishu.cn/wiki/KKnrwAmu1idS1dkqqLHcknGWnZd?from=from_copylink}}

\begin{itemize}[itemsep=1ex, topsep=1ex, parsep=0.5ex, leftmargin=1em]
  \item 为提升 Lenovo 业务数据分析效率与质量,在决策层方面提供数据分析。核心方法是基于检索和多视角主题大纲合成,从不同主题视角综合汇总,形成全面、深入的分析报告;技术实现上,选用开源项目 storm 作为开发框架,结合本地部署的 qwen-72b 模型及检索器,对检索信息进行深度分析和整合。
\end{itemize}



% Reference Test
%\datedsubsection{\textbf{Paper Title\cite{zaharia2012resilient}}}{May. 2015}
%An xxx optimized for xxx\cite{verma2015large}
%\begin{itemize}
%  \item main contribution
%\end{itemize}

\section{\faCogs\ 个人技能}
% increase linespacing [parsep=0.5ex]
\begin{itemize}[parsep=0.5ex]
  \item \textbf{专业技能:} 熟悉 Python 语言、PyTorch 框架、Linux、Docker、Git、开源平台 Hugging Face 、GitHub(如 Llama-Factory、vLLM、SGLang、及其他优秀开源项目等)
  \item \textbf{英语技能:}大学英语四/六级 CET-4/6,具备良好的英文文献检索及阅读能力
\end{itemize}


\section{\faInfo\ 其他}
% increase linespacing [parsep=0.5ex]
\begin{itemize}[parsep=0.5ex]
  \item \textbf{智能领域:}具备深度学习、AIGC 相关知识,熟悉 NLP、CV 基础理论和算法,Transformer 系列模型;具有 SFT、RL 优化经验。
  \item \textbf{团队协作:}有很强的责任心和团体协作能力。工作中经常分享 AI 相关论文、技术报告等,在部门组会中分享过 DeepSeek-R1、Kimi-1.5、Gemini-2.5-Pro、CoT 技术及相关工作等。

\end{itemize}

%% Reference
%\newpage
%\bibliographystyle{IEEETran}
%\bibliography{mycite}
\end{document}
